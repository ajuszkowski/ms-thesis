\chapter{The \texttt{mousquetaires}: implementation}
\label{ch:impl}

This chapter describes the architecture of the tool \texttt{mousquitaires} ...

language: java

% why: https://stackoverflow.com/a/13968707/4977823

\section{Program Requirements}
\label{ch:impl:requirements}

- stability (tests)

- scalability (new features of language, new models, new tasks for a program )

- transparency

- efficiency

\section{Program Components}
\label{ch:impl:comp}
Big view

\section{C11 to YTree parser}
\label{ch:impl:ytree}

below: mostly mock text.

- The language-dependent syntax tree:
        - for now it's the C subset language which I called 'Cmin'; as a base, I used the C11 grammar from ANTLR github repository, then I simplified it a lot, cutting off many unnecessary C syntax features and making it more convenient for parsing. When developing the Cmin language, I kept in mind C elements that are necessary for processing the linux kernel code, though for now not the whole grammar element described in file 'Cmin.g4' are being implemented;
        - later I am going to add the litmus grammar as well;
        - in future, it will be not a problem to add any new C-like language;

- The language-independent abstract syntax tree (aliased 'Ytree', where 'Y' resembles branching of the tree):
        - all tree nodes in my code are prefixed with 'Y', see tentative (yet almost complete) class hierarchy in picture 'YEntity.png';
        - this AST contains very basic language elements according to the C execution model (statements and expressions);
        - converting the language-dependent syntax tree to the language-independent syntax tree is performed by Visitor pattern (e.g., for Cmin->Ytree conversion is made by 'CminToYtreeConverterVisitor')
        - minor changes are performed by converting to ytree representation: desugaring the target code, etc.


\section{YTree to XGraph event converter}
\label{ch:impl:x2y}

- Then, the AST is being interpreted and converted to event-based representation (aliased 'Xrepr' for eXecution representation):
        - more low-level code representation (or high-level assembly);
        - I try to keep this representation close to the one you described in your papers: basic load \& store events, branching events, fence events;
        - this representation is being implementing these days, I've just started doing it (see current class hierarchy in the picture 'XEntity.png');
        
- After we acquired the event-based representation, we can perform some modifications/simplifications/optimisations on it (separately, allowing user to manage them):
        - converting to SSA form as one of necessary steps before encoding;
        - (more? -- I'm not thinking about it yet);
        
\subsection{Loop unrolling}
\label{ch:impl:x2y:unrolling}
% If the analysing program contains a loop, then its control-flow graph represents the cyclic directed graph. Although, in order to be able to encode such a cyclic structure as a boolean formula, this graph needs to be  
% In order to be

The original program encoded into the \texttt{XGraph} represents a \textit{flow graph}, a connected cyclic directed graph with single source node \texttt{[ENTRY]} (usually for convenience all leaves are connected to the sink node \texttt{[EXIT]}). The cycles are caused by low-level jump instructions, obtained from non-linear high-level control-flow statements (such as \texttt{while}, \texttt{do-while}, \texttt{for}, etc.). However, the cyclic flow graph cannot be encoded into SMT formula since ...
//TODO:REFERENCE.%TODO



\begin{figure}
\begin{tikzpicture}[->,>=stealth',shorten >=1pt,auto,node distance=1.5cm,semithick]
\node[c] (1) [] {$1$};
\node[c] (2) [below left of=1] {$2$};
\node[c] (3) [below right of=1] {$3$};
\node[c] (4) [below of=3] {$4$};
\path[->]
(1) edge [] node {} (2)
(2) edge [bend left=50,dotted] node {} (1)
(1) edge [] node {} (3)
(3) edge [] node {} (4)
(4) edge [bend right=80,dotted] node {} (1)
;
\node[draw=none] (impl) [right=3cm of 3] {$\overset{k = 6}{\longmapsto}$};
;
\node[c] (21) [right=3cm of impl] {$2_1$};
\node[c] (11) [above right=1cm and 1cm of 21]{$1_1$};
\node[c] (31) [below right=1cm and 1cm of 11] {$3_1$};
\node[c] (41) [below of=31] {$4_1$};
\node[c] (12) [below of=21] {$1_2$};
\node[c] (22) [below left=1cm and 1cm of 12] {$2_2$};
\node[c] (32) [below right=1cm and 1cm of 12] {$3_2$};
\node[c] (13) [below of=41] {$1_3$};
\node[c] (14) [below of=22] {$1_4$};
\node[c] (43) [below of=32] {$4_3$};
\node[c] (23) [below of=13] {$2_3$};
\node[c] (33) [below of=13] {$3_3$};
\node[c] (24) [below left=1cm and 1cm of 14] {$2_4$};
\node[c] (34) [below right=1cm and 1cm of 14] {$3_4$};
\node[c] (15) [below right=1cm and -0.3cm of 43] {$1_5$};
\node[c] (44) [below of=33] {$4_4$};
\node[c] (6) [below left=1cm and 1cm of 15] {$6$};
\node[] (11k) [right=3.2cm of 11] {$(k = 1)$};
\node[] (31k) [right=1.7cm of 31] {$(k = 2)$};
\node[] (41k) [right=1.7cm of 41] {$(k = 3)$};
\node[] (13k) [right=1.7cm of 13] {$(k = 4)$};
\node[] (33k) [right=1.7cm of 33] {$(k = 5)$};
\node[] (44k) [right=1.7cm of 44] {$(k = 6)$};
\path[->]
(11) edge [] node {} (21)
(11) edge [] node {} (31)
(31) edge [] node {} (41)
(21) edge [] node {} (12)
(12) edge [bend right=10] node {} (22)
(12) edge [bend left=10] node {} (32)
(41) edge [] node {} (13)
(22) edge [] node {} (14)
(32) edge [] node {} (43)
(13) edge [] node {} (23)
(13) edge [] node {} (33)
(14) edge [bend right=10] node {} (24)
(14) edge [bend left=10] node {} (34)
(43) edge [] node {} (15)
(33) edge [bend right=10] node {} (15)
(33) edge [bend left=10] node {} (44)
(24) edge [bend right=20] node {} (6)
(34) edge [] node {} (6)
(15) edge [] node {} (6)
(44) edge [bend left=20] node {} (6)
;
\end{tikzpicture}
\label{fig:loop-unwind}
\caption{Example of the flow graph from the Figure~\ref{fig:merged-loop.....}, unwinded up to the bound $k = 6$} 
\end{figure}


\section{XGraph to ZFormula (SMT) encoder}
\label{ch:impl:comp:zformula}

- Then, this modified event-representation is being encoded to SMT formula and sent to the solver.



\section{Optimisations}

... performed on each stage