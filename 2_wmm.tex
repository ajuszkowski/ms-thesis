\chapter{Memory model-aware analysis}
\label{ch:wmm}

The main idea behind the memory model-aware program analysis is that the set of all possible executions of the concurrent program (the \textit{anarchic semantics}) can be specified by the axiomatic constraints of the memory model that filter out executions inconsistent in particular architecture (the \textit{analytic semantics})~\cite{alglave2016syntax}.
The anarchic semantics of the program is a truly parallel semantics with no global time that describes all possible computations with all possible communications.
However, the analytic semantics captures the program behaviours on a certain execution environment more precisely.
%"The semantics of a program is a set of executions" https://johnwickerson.github.io/papers/memalloy.pdf


\section{Event-based program representation}
\label{ch:wmm:event}

The classical approach for analysing concurrent programs is to model it as the set of sequentially consistent programs, obtained by enumerating all possible interleavings.
These models are deterministic as they include the notion of the \textit{global time}. %, a single order of interleavings among all events happened in different threads.
Although these models are easy to build and analyse, the number of all possible interleavings grows exponentially (known as the \textit{combinatorial explosion}), which affects the completeness of an analysis method in general case.

One way to fight the combinatorial explosion is to exclude the global time from the model and treat executions from one equivalence class together in a non-deterministic fashion.
For instance, such an equivalence class can be the set of computations performed by a processor locally that do not affect the global state.
This idea is used in the \textit{event-based} model, that represents the program as a directed graph of events (the \textit{event-flow graph})~\cite{alglave2010shared,alglave2014herding}.
The vertices of such a graph represent \textit{events} (see Section~\ref{ch:wmm:model:events}), and edges represent basic relations (see Section~\ref{ch:wmm:model:relations}).
The graph represents the set of executions (sequences of events; see Section~\ref{ch:wmm:model:executions}) defined by the non-deterministic guesses of certain relations on some states.

There are three main types of sources of non-determinism in concurrent programs~\cite{musuvathi2008fair}:
\begin{enumerate}[noitemsep,topsep=0pt]
\item \textit{input non-determinism}, which is a standard undecidable problem for all static analysis methods: to resolve the user input, system call from the environment, unresolved function calls, etc.;
\item \textit{scheduling non-determinism}, caused by the interleavings, which in turn are caused by the scheduler activity; and
\item \textit{memory-model non-determinism}, caused by hardware and compiler relaxations.
\end{enumerate}

The event-based program model is able to emulate effectively the second and the third types of sources of non-determinism, while the first one can be coped by standard static analysis methods~\cite{landi1992undecidability,SurveySymExec-CSUR18}.


\subsection{Events}
\label{ch:wmm:model:events}

An \textit{event} is a fact of executing the low-level primitive atomic operation such as memory access, threads synchronisation, computation over the local-memory, control-flow jump, etc.

A \textit{memory event} $e_m \in \mathbb{E}$ represents the fact of access to the memory.
Only memory events change the state of an abstract machine executing the code, since it is completely determined by values stored in its memory.
Since memory is the crucial low-level resource shared by multiple processes, most relations are defined over memory events. 
The processes can access a shared memory location (denoted by~$l_i$, for \textit{location}), or a local one (denoted by~$r_i$, for \textit{register}).
A memory event can access at most one shared memory location, high-level instructions that address more than one shared variable must be transformed into a sequence of events.
A memory event is specified by its direction with respect to the shared variable, its location~$\texttt{loc}(e_m)$, its processor label~$\texttt{proc}(e_m)$, and a unique event label~$\texttt{id}(e_m)$~\cite{alglave2010shared}.
%\texttt{load} for read the value of a shared-memory location, or \texttt{store} for write, or neither of them if both locations are local

The set of memory events $\mathbb{M}$ is divided into write events $\mathbb{W}$ (that write values to shared-memory locations) and read events $\mathbb{R}$ (that read values stored in shared-memory locations).
We add a restriction that each memory event uses at most one shared location, so that the write instruction $i = write(l_1, l_2)$, that encodes the write from the shared location $l_2$ to the shared location $l_1$, is represented as two consequent events $e_1~=~\texttt{load}(r_1~\leftarrow~l_2); \ e_2~=~\texttt{store}(l_1~\leftarrow~r_1)$.
Also, it is important to separate the set of initial write events $\mathbb{IW}~\subseteq~\mathbb{W}$ that perform initialisation of program variables.

A \textit{computation event} $e_c \in \mathbb{C} \subseteq \mathbb{E}$, represents a low-level assembly computation operation performed solely on local-memory arguments.
An example of computation event may be the event $e_c = r_1 \leftarrow add(r_2, 1)$ that writes the sum of values stored in register $r_2$ and constant $1$ (which is modelled as a register as well) to the register $r_1$.
For modelling branching statements, we define the set $\mathbb{C}_{g}~\subseteq~\mathbb{C}$ of \textit{guard} computation events (also called as \textit{branching events}), that are evaluated to a boolean value.

Synchronisation instructions (fences) cause \textit{barrier events}, which do not perform any computation or memory value transfer, instead, they the set of program behaviours by adding barrier relations to the program model.
Functionally, a fence may be a synchronisation barrier or a instruction for flushing memory caches into the main memory, etc.
For instance, the \texttt{mfence} instruction of the x86 assembly flushes the store buffers of the thread, and thus does not allow the \rf-relation (see Section~\ref{ch:wmm:model:relations}) to point to 


\subsection{Relations}
\label{ch:wmm:model:relations}

%In this section, we describe basic relations used in memory model-aware program analysis.

The relation~$r\,\subseteq~\mathbb{E}~\times~\mathbb{E}$ is a set of pairs of events (a subset of Cartesian product of two sets of events). There are two kinds of relations between events: \textit{basic relations} that capture the semantics of the program, and \textit{derived relations} that are defined from the basic relations and events in the weak memory model specification. Constraints over relations that are specified by weak memory models are defined as requirements of \textit{acyclicity}, \textit{irreflexivity} or \textit{emptiness} of specific relations~\cite{alglave2016syntax}.

\vspace{1em}
The basic relations are the following~\cite{alglave2010shared}:
\begin{itemize}
    \item The \textit{control-flow} of a program is defined by the \textit{program-order} relation \po~$\subseteq~\mathbb{E}~\times~\mathbb{E}$, which represents the total order of events of same process.
    For instance, if the instruction $i_1$ generates the event $e_1$ and the instruction $i_2$ follows $i_1$ and generates the event $e_2$, then $e_1 \relarrow{po} e_2$.
    %Some new relations may be acquired : dp, po-loc

    \item The \textit{data-flow} of a program is defined by \textit{communication relations}:
    \begin{itemize}[noitemsep]
        \item the \textit{read-from} relation \rf{}~$\subseteq~\mathbb{W}~\times~\mathbb{R}$ that maps each write event to the read event that reads the value written by write event; and
        \item the \textit{coherence-order} relation \co{}~$\subseteq~\mathbb{W}~\times~\mathbb{W}$ that defines the total order on writes to the same location across all processes (also called the \textit{write serialisation}, \texttt{ws}-relation).
    \end{itemize}

    \item Events from the same process are related by the \textit{scope relation} \sr{}~$\subseteq~\mathbb{E}~\times~\mathbb{E}$.
    In contrast to the \tool{herd} tool, \porthos[2] does not use hierarchy of scopes (depicted as the scope tree); instead, it uses simple labels that indicate which process has produced certain event.
\end{itemize}

%TO-DO: add ppo = characteristics of memory model. "Architectures are instances of our model. An architecture is a triple of functions (ppo, fences, prop), which specifies the preserved program order ppo, the fences fences and the propagation order prop." (herding cats, p.13)

Below we enumerate some derived relations~\cite{alglave2010shared}:

\begin{itemize}

    \item the \textit{from-read} relation $\fr \subseteq~\mathbb{R}~\times~\mathbb{W}$ that maps a read event to all write events succeeding the write event from which the read event gets its value: \\
    $r \relarrow{fr} w ~\triangleq~ (\exists w' . w' \relarrow{rf} r \land w' \relarrow{co} w)$;

    %\item the \textit{data dependency} relation \texttt{dp}, which is a subset of \po-relation that always has a read at its source (it connects the read to the write which it depends on)
    %TO-DO: formula for dp or exclude

    \item the \textit{communication} relation \com over memory events, that fully describes the data-flow of a program: \\
    %$\texttt{com} \triangleq (\texttt{rf} \cup \texttt{co} \cup \texttt{fr})$
    $m_1 \relarrow{com} m_2~\triangleq~((m_1\relarrow{rf}m_2) \lor (m_1\relarrow{co}m_2) \lor (m_1\relarrow{fr}m_2))$;

    \item the \textit{external} (and \textit{internal}) \textit{from-read} relations that restrict the \fr-relation to the different (respectively, same) processes: \\
    $w \relarrow{fre} r ~\triangleq~ (w \relarrow{fr} r \land \texttt{proc}(w) = \texttt{proc}(r))$, \\
    $w \relarrow{fri} r ~\triangleq~ (w \relarrow{fr} r \land \texttt{proc}(w) \neq \texttt{proc}(r))$;
    
    %TO-DO: say that po-relation is *static*
    \item the \texttt{po-loc} relation that is the \po-relation over events that access to the same shared variable: \\
    $m_1 \relarrow{po-loc} m_2 ~\triangleq~ (m_1 \relarrow{po} m_2 \land \texttt{loc}(m_1) = \texttt{loc}(m_2))$; and

    \item the semantics of \textit{fences} (memory barriers) specific for different architectures may be defined as derived relations.

    %"Preserved program order defines the set of reorderings guaranteed by the architecture not to occur, based on the types of the accesses." http://delivery.acm.org/10.1145/2750000/2742219/p635-lustig.pdf

\end{itemize}


\subsection{Executions}
\label{ch:wmm:model:executions}

The semantics of a concurrent program is represented as the set of allowed executions.
An \textit{execution} is a path in the event-flow graph defined by \po- and \rf-relations and set of final writes to a given memory location that is valid under certain memory model~\cite{alglave2014herding}.
It can be interpreted as a sequence of guesses which event is to be executed next.
A \textit{candidate execution} is an execution that is not yet constrained by a memory model.

%An \textit{execution} (trace, run) of a program is an ordered set of events defined by \po- and \rf-relations and set of final writes to a given memory location that is valid under certain memory model~\cite{alglave2014herding}.
Figure~\ref{simple_wmm_x86_pic} illustrates four possible candidate executions for the litmus test Example~\ref{simple_wmm_x86} (the pictures are generated by the \tool{herd7} tool, version~7.47).
Since there are no conditional jumps, the \po-relation is defined and we do not need to guess it.
Since each thread performs a single write followed by a single read, the \co-relation is also defined (it relates the initial write event with the write event to the same location).

\begin{figure}[!htb]
\centering
\begin{subfigure}[t]{.28\textwidth}
  \centering
  \includegraphics[width=1.2\linewidth]{img/my/sb-example/SB-dot.png}
  \caption{Final state: \texttt{(0:EAX=1,~1:EAX=1)}}
  \label{simple_wmm_x86_pic:sub1}
\end{subfigure}
\hfill
\begin{subfigure}[t]{.23\textwidth}
  \centering
  \includegraphics[width=.9\linewidth]{img/my/sb-example/SB-dot-2.png}
  \caption{Final state: \texttt{(0:EAX=1,~1:EAX=0)}}
  \label{simple_wmm_x86_pic:sub2}
\end{subfigure}
\hfill
\begin{subfigure}[t]{.23\textwidth}
  \centering
  \includegraphics[width=.9\linewidth]{img/my/sb-example/SB-dot-3.png}
  \caption{Final state: \texttt{(0:EAX=1,~1:EAX=1)}}
  \label{simple_wmm_x86_pic:sub3}
\end{subfigure}
\hfill
\begin{subfigure}[t]{.23\textwidth}
  \centering
  \includegraphics[width=.9\linewidth]{img/my/sb-example/SB-dot-4.png}
  \caption{Final state: \texttt{(0:EAX=0,~1:EAX=0)}}
  \label{simple_wmm_x86_pic:sub4}
\end{subfigure}
\hfill
\caption{Candidate executions for the litmus test in Example~\ref{simple_wmm_x86}}
\label{simple_wmm_x86_pic}
\end{figure}

Thus, there are only four possible executions defined by the choice of \rf-relation.
The candidate executions pictured in Figures~\ref{simple_wmm_x86_pic:sub1}--\ref{simple_wmm_x86_pic:sub3} are consistent both under strong memory model SC and under relaxed memory models x86-TSO, Power, ARM, and some others.
However, the execution shown in Figure~\ref{simple_wmm_x86_pic:sub3} is still consistent under relaxed-memory architectures, but it becomes inconsistent under SC architecture as it forbids cycles over~$\fr\cup\po$.
%However, the Power memory model allows such cycles, therefore %TODO: include load buffering example, see presentation pdf


\section{The \cat{} language}
\label{ch:wmm:cat}

Weak memory models are defined via \cat{} language~\cite{alglave2016syntax}.
It is a domain specific language for describing consistency properties of concurrent programs.
The language combines expressive power of a functional language (being inspired by OCaml, it adopts the OCaml types, first-class functions, pattern matching and some other features) with concepts of memory models (sets of events, relations, operations and assertions over relations).
In \cat{}, new relations can be defined via the keyword \texttt{let} and the following operators over relations~\cite{alglave2016syntax}.

Below we enumerate pre-defined operations over relations and sets of events:

\begin{enumerate}
  \item \textit{Unary operations}:
    \begin{itemize}
      \item the \textit{complement} of a relation \texttt{r} is $\texttt{\textasciitilde r}$,
      \item the \textit{transitive closure} of a relation \texttt{r} is $\texttt{r+}$,
      \item the \textit{reflexive closure} of a relation \texttt{r} is $\texttt{r?}$,
      \item the \textit{reflexive-transitive closure} of a relation \texttt{r} is $\texttt{r*}$, and
      \item the \textit{inverse} of a relation \texttt{r} is $\invrel{r}$.
    \end{itemize}
  \item \textit{Binary operations:}
  \begin{itemize}
    \item the \textit{union} of two relations \texttt{r1} and \texttt{r2} is \texttt{r1\,|\,r2},
    \item the \textit{intersection} of two relations \texttt{r1} and \texttt{r2} is \texttt{r1\,\&\,r2},
      \item the \textit{difference} of two relations \texttt{r1} and \texttt{r2} is $\texttt{r1} \relminus \texttt{r2}$, and
    \item \textit{the sequence} of two relations \texttt{r1} and \texttt{r2} is \texttt{r1;r2}, which is defined as the set of pairs \texttt{(x,y)} such that there exists an intervening \texttt{z}, such that $\texttt{(x,z)} \in \texttt{r1}$ and $\texttt{(z,y)} \in \texttt{r2}$.
  \end{itemize}
\end{enumerate}

For instance, the \fr-relation is defined as a sequence of inverted \rf-relation and \co-relation: $\texttt{fr} = (\invrel{rf};\texttt{co})$.
As an example of memory model definition in \cat{} language, Figure~\ref{ex:x86-cat} presents the excerpt from the x86-TSO memory model~\cite{herd10tutorial}.
This memory model specification asserts  acyclicity of the communication relation (the union of \rf-, \fr- and \co- relations), \texttt{po-loc} -relation, \texttt{mfence}-relation and some other derived relations~\cite{owens2009better}.

\begin{figure}[H]
\begin{lstlisting}
...
let po_ghb = WW(po) | RM(po)
let implied = PA(poWR) | WR(po)
let GHB = mfence | implied | po_ghb | rfe | fr | co
let com = rf | fr | co

empty atom & (fre;coe)
acyclic po-loc | com
acyclic GHB
\end{lstlisting}
\caption{Excerpt from the x86-TSO memory model in the \cat{} language}
\label{ex:x86-cat}
\end{figure}
