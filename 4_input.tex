\chapter{The input language}
\label{ch:impl}

The \Porthos tool used a small subset of C as an input language as described on Figure~\ref{syntax:in_grammar_pts}~\cite{Porthos17}:

\begin{figure}[H]
%\begin{multicols}{2}
\begin{lstlisting}[mathescape=true,
        caption={Syntax of an input language of \Porthos},
        label={syntax:in_grammar_pts},
        literate={<--}{{$\leftarrow$}}1,
        morekeywords={if,then,else,return,while,program,thread,mfence,sync,lwsync,isync}
        breaklines=true,
        ]
<prog> : program <thrd>*
       ;
<thrd> : thread <tid> <inst>
       ;
<inst> : <atom> 
       | <inst> ; <inst>
       | while <pred> <inst>
       | if <pred > then <inst> else <inst>
       ;
<atom> : <reg> <-- <exp> 
       | <reg> <-- <loc>
       | <loc> := <reg> 
       | mfence
       | sync
       | lwsync
       | isync
       ;
\end{lstlisting}
%\end{multicols}
\end{figure}


The input language of \Porthos supported recursive definitions of control-flow instructions, that were directly encoded into SMT-formula. In \mousquetaires, the set of supported instructions is extended in order to support expressions valid in C.%TODO: examples, which  we extended
 Also, for the purpose of simplicity, all kinds of fence instructions and different types of variable assignments (shared variable assignment~`\texttt{:=}' or local variable assignment~`$\mathtt{\leftarrow}$') were defined directly in the grammar for \Porthos, whereas the semantics of function invocations need to be determined dynamically during the interpretation
% according to the set of known functions
(see Chapter~\ref{???}).

Both \Porthos and \mousquetaires \ use the parser generator ANTLR~\cite{parr2013definitive}, the powerfull language processing tool. The (hand-written) ANTLR grammar used in \Porthos is available at Appendix~\ref{apx:in_grammar_pts}, whereas the \mousquetaires uses the grammar of C language proposed in the C11 standard~\cite{jtc2011sc22} (the ANTLR grammar can be found in the official ANTLR repository on GitHub~\footnote{\url{https://github.com/antlr/grammars-v4}}).
