\chapter{The input language}
\label{ch:impl}

%TODO: say sth about 100500 litmus-tests for kernel

The previous version of \porthos processed the small subset of C as an input language (see grammar in Figure~\ref{syntax:in_grammar_pts})~\cite{Porthos17}.
It supported recursive definitions of control-flow instructions, that were directly encoded into SMT-formula.
The grammar supported by \porthos[2] was extended in order to support expressions valid in C. %TODO: examples, how imenno extended
Also, the input language for the first version of \porthos  
Also, for the purpose of simplicity, all kinds of fence instructions and different types of variable assignments (shared variable assignment~`\texttt{:=}' or local variable assignment~`$\mathtt{\leftarrow}$') were defined directly in the grammar for \porthos, whereas the semantics of function invocations need to be determined dynamically during the interpretation
% according to the set of known functions
(see Chapter~\ref{???}).


\begin{figure}%[H]
%\begin{multicols}{2}
\begin{lstlisting}[mathescape=true,
        caption={Syntax of an input language of \porthos},
        label={syntax:in_grammar_pts},
        literate={<--}{{$\leftarrow$}}1,
        morekeywords={if,then,else,return,while,program,thread,mfence,sync,lwsync,isync}
        breaklines=true,
        basicstyle=\ttfamily\scriptsize,
        ]
<prog> : program <thrd>*
       ;
<thrd> : thread <tid> <inst>
       ;
<inst> : <atom> 
       | <inst> ; <inst>
       | while <pred> <inst>
       | if <pred > then <inst> else <inst>
       ;
<atom> : <reg> <-- <exp> 
       | <reg> <-- <loc>
       | <loc> := <reg> 
       | mfence
       | sync
       | lwsync
       | isync
       ;
\end{lstlisting}
%\end{multicols}
\end{figure}


Both \porthos and \porthos[2] \ use the parser generator ANTLR~\cite{parr2013definitive}, the powerfull language processing tool. The (hand-written) ANTLR grammar used in \porthos is available at Appendix~\ref{apx:in_grammar_pts}, whereas the \porthos[2] uses the grammar of C language proposed in the C11 standard~\cite{jtc2011sc22} (the ANTLR grammar can be found in the official ANTLR repository on GitHub~\footnote{\url{https://github.com/antlr/grammars-v4}}).
