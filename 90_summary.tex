\chapter{Summary}
\label{ch:summary}

%instead of extending the \textit{grammar} of input language, we extended the \textit{architecture} of the tool so that the full C language can be supported

\section{Solved tasks and contributions}

During the work on this thesis, we solved the following research and engineering problems:

%TODO: written a generic hardware-agnostic extensible non-optimising compiler

\begin{enumerate}[label=\arabic*.] %[label=\Roman*.]
\item
We have studied existing weak memory model-aware analysis approaches, tools and frameworks to gather deep understanding of the problem (Sections~\ref{ch:intro:related} and~\ref{ch:enc:mc} and Chapter~\ref{ch:wmm}).

\item
Then, we explored existing implementation of the portability analysis tool \textit{\porthos{}} and designed the new tool \textit{\porthos[2]} that accepts as input the larger subset of the C language and supports its fundamental concepts by design, has modular architecture and disposes the extensible knowledge base of the domain-specific functions.
The key aspects of the architectural design and implementations are the following:

  \begin{enumerate}[label=\alph*.]
  \item The first stumbling block in extension of the input language was the \porthos[1] program input parser, which performs the full semantic analysis of the program.
  Although it works for a small subset of the C language, the rich and expressive language such as C requires several stages of analysis before the compilation stage.
  To handle this, we implemented the processing units that analyse the AST of the program on the \textit{pre-compilation} stage (see Section~\ref{ch:impl:proc:x-pre-compiler}).

  As an example, the \textit{variable kind analysis} (determining whether they are shared or local) is performed by \porthos[1] syntactically, depending on the operator or the function that uses the variable.
  In contrast, \porthos[2] traverses the program AST and collects information of all variables accessed in the program during the pre-compilation stage (see Section~\ref{ch:impl:proc:x-pre-compiler}).
  Currently, \porthos[2] considers a variable to be shared if it was declared as a pointer, or its address is accessed at least once by the code of any process, or it was declared as a parameter of the function that defines a process, or it was exported by \texttt{extern} keyword or an exporting function.

  \item In addition, \porthos[2] supports \textit{new syntactic constructions} of the C language: \texttt{break} and \texttt{continue} jumps, function invocations, multiple-variable declarations (as `\lstinline{int a, b=2, z;}'), arbitrary expressions allowed by the C11 standard~\cite{c11}, and some others (see Section~\ref{ch:impl:input}).

  \item As the C language supports unconditional \texttt{goto}-jumps, the control-flow graph can have an arbitrary structure, which can not be modelled solely by the program AST, therefore \porthos[2] compiles the AST to the low-level hardware-agnostic program representation \xgraph{}.
  For that, we implemented the \textit{full C~interpreter} discussed in Section~\ref{ch:impl:proc:x-compiler}.
  
  \item Originally, the control-flow instructions were encoded directly into the SMT-formula~\cite{Porthos17a}.
  In contrast, \porthos[2] encodes the low-level \xgraph{} representation in accordance with the \textit{new control-flow encoding scheme} that in general follows the one proposed in~\cite[Chapter 5.1.2]{heljanko2008unfoldings} (see Section~\ref{ch:enc:bmc:cf}).
  As the new encoding scheme does not add new variables to every control-flow instruction, the number of variables in the result SMT-formula is expected to be smaller.
  
  \item Since \porthos[2] compiles the program AST to the \xgraph{} before encoding it into the SMT-formula, we decided to change the \textit{unrolling algorithm} from unwinding all loops $k$ times (where $k$ is the user-specified unrolling bound) to the DFS-based algorithm that explores \textit{all possible executions} that the program can produce within $k$ steps (see discussion on the unrolling in Sections~\ref{ch:impl:proc:x-unroll} and~\ref{ch:eval:show:compil}).
  On the other hand, it is shown in Section~\ref{ch:eval:show:unrol} that the implementation of the new unrolling algorithm does not produce exponentially many dummy \texttt{nop}-events as the implementation of the old unrolling algorithm does.
  Another benefit of the new unrolling scheme is that it is much more configurable (i.e., the number of events grows gradually as the unrolling bound is being increased, see Figure~\ref{dep:events}). \\
  Although the new unrolling algorithm produces considerably more executions than the old one and thus requires more time for analysis, \porthos[2] uses the new unrolling algorithm by default as it is complete (finds all possible executions).

  \item For ease of adding support for new domain-specific functions (for example, the Kernel-specific atomic write function \texttt{atomic\_store}), we implemented the \textit{invocation hooking}, a flexible mechanism for intercepting the compilation process without changing the interpreter.
  The invocation hooking mechanism serves as a knowledge base for the program domain, that is to be filled and extended in future.
  Invocation hooks are defined in Java and thus are flexible, though their extension and modification requires some knowledge of the internals of the tool.
  We illustrate this mechanism with the basic support for Linux kernel litmus tests (see Sections~\ref{ch:impl:proc:x-compiler} and~\ref{ch:eval:perf:feat}).

  \end{enumerate}
  
\item 
As the tests show, the \textit{performance overhead} of the new architecture is negligible (see Section~\ref{ch:eval:perf}), therefore we consider the applied architectural decisions to be acceptable.

\end{enumerate}


\section{Limitations and directions for future work}

Current implementation of \porthos[2] has the following limitations, that might possibly define the direction of the future work.

\begin{itemize}[leftmargin=\parindent]
\item One of the major limitation of \porthos[2] as a software verification tool is its sensitivity to the \textit{combinatorial explosion} of the state space.
%that performs complete analysis
As it was shown in Section~\ref{ch:eval:perf}, the number of events of the unrolled program grow rapidly as the user increases the unrolling bound.
One possible way to reduce the number of the program states might be applying some traditional techniques that target the state explosion problem (such as concrete execution as a part of \textit{concolic execution}~\cite{majumdar2007hybrid} before the unrolling stage).
However, this must be done carefully and with taking into account the analysing weak memory model, because otherwise it can lead to the loss of states and thus to incorrect analysis results.
Nonetheless, the small-size litmus test-like programs (containing hundreds of events after the unrolling) remain to be handleable by \porthos[2] within reasonable time.
% cannot be applied to our problem \textit{before} the weak memory model is considered, 
%One possible enhancement, that could treat but not cure this flaw, could be an extra analysis stage carried before the unrolling, that analyses 

%\item As its predecessor, \porthos[2] does not propose any heuristic for finding the optimal unrolling bound.

\item For justifying the correctness of the verification that bases on the result of the input program unrolling (via the new unrolling scheme discussed in Section~\ref{ch:impl:proc:x-unroll}), one might want to be aware whether the execution in the unrolled event-flow graph has run the loop a whole number or it was interrupted on non-last event of the loop.
This can be modelled with \textit{two kinds of sink events} (completed and uncompleted) of the event-flow graph as it was discussed in Sections~\ref{ch:enc:bmc:cf} and~\ref{ch:impl:proc:x-unroll}.

\item \textit{Extending the knowledge base} of domain-specific functions to model synchronisation primitives (the compiler mapping) can be considered as the main future research direction.
Thus, to be able to process most Linux kernel litmus tests, \porthos[2] needs to know the semantics of the barrier and memory operation functions that are specific for the Linux kernel in order to be able to compile them into the \xgraph{} and later encode them into the SMT-formula.
Due to modular architecture of \porthos[2], the extension of the knowledge base can be done by modifying solely the invocation hooking mechanism.
Note that the flip side of the flexibility of invocation hooks is their complexity: being written in Java, in addition they require from the user some knowledge of general architecture of the tool, the interface methods provided by the interpreter, the class hierarchy of the \xgraph{} elements, etc.

\item The wide range of existing litmus tests may require \porthos[2] to \textit{support new input languages}.
The first candidates for being supported are the LISA language (Litmus Instruction Set Architecture)~\cite{alglave2016syntax} and assembly languages of most common architectures (x86, Power, Alpha, etc.).
Note that the programs in low-level languages can be easily compiled to the \xgraph{} representation by existing compiler architecture since low-level assembly-like languages can be considered as the subset of the C language, that restricts the set of non-linear control-flow instructions to the conditional and unconditional jumps (which are supported by the current \xgraph{} compiler of \porthos[2]).
%Note that adding support for low-level languages requires adding the new \xgraph{} compiler that will compile the low-level syntax tree (with non-conditional jumps as the only )

\item Although \porthos[2] has an extended support for primitive data types (integers, reals, booleans), it still is not able to handle the \textit{complex data types}.
However, as the \tool{Z3} solver supports the theory of constant-size arrays, it might be relatively easy to extend the support for constant-size arrays (declared as `\lstinline{int arr[10];}'), enumerations and structures.
Nevertheless, pointers and variable-size arrays (allocated in the heap by functions \lstinline{malloc}, \lstinline{calloc}, etc.) require a stronger pre-processing analysis before being encoded into an SMT-formula.

\item Currently, \porthos[2] can operate only in the intra-procedural analysis mode by assuming that each function provided by the user represents a separate thread (so-called litmus test-mode).
One future direction of the work could be adding the \textit{inter-procedural analysis mode} for scanning a code project instead of a separate file (see the discussion in Section~\ref{ch:impl:principles}).
Although, processing large programs may require serious memory optimisations (for instance, in storing the full unrolled graph data structure), which can possibly lead to worsening the overall performance.
In addition, when processing large code projects, \porthos[2] might need the memory-guarding module that tracks the memory consumption of the tool and aborts its work if necessary.

\item As it was mentioned in Section~\ref{ch:enc:bmc:cf} devoted to the encoding for the control-flow of the program, the new encoding scheme used by \porthos[2] allows to analyse \textit{partial functions}.
Although the current \porthos[2] interpreter infrastructure is not configured to perform a partial analysis, it can be easily supported.

\item During the implementation of \porthos[2], we have only done limited \textit{testing of the tool}.
Thus, in order to increase the stability of \porthos[2], we need to cover its code by unit- and functional tests.

\item One might want to compare the performance of \porthos[2] in cases of usage the different SMT-solvers.
For that, the \zformula{} abstraction layer will be useful as new SMT-solvers can be easily added (the general way to support multiple SMT-solvers is to convert the \zformula{} to an SMT-LIB formula~\cite{smt-lib}, that can be passed as input to the SMT-solver run as an external process).

%\item <parallelised>

%\item <C processor directives + typedef > 
%<TODO: a short summary here>

%\item <contexts for local mem units in mem manager>

%\item <as an optimisation, maybe use more comprehensive unrolling algo as in Aho>

%\item In order to be able to assess the A user might want to know

%\item optionally: urnolling algo; 

\end{itemize}


\ifdefined\IsAalto
\else
  \vspace*{1em}
\fi


The new tool \porthos[2] with the extensible program domain knowledge base can be considered as a generalised framework for SMT-based memory model-aware analysis, which can not only perform the reachability and portability analysis, but also serve as a basis for other kinds of static analysis of concurrent programs.
