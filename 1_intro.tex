\chapter{Introduction}
\label{ch:intro}

%TODO: rename mousquetaires: http://www.cprover.org/wmm/musketeer/ !!! (+reference this project: "Don’t sit on the fence".pdf in alglave/ )
% TODO: see how musketeer deals with recursive calls: http://www.cprover.org/wmm/musketeer/rec.pdf

\section{Problem statement}
\label{ch:intro:problem}

Most modern computer systems contain large parts that operate concurrently.
Although the parallelisation of a system can drastically improve its performance, it opens numerous of problems regarding correctness, robustness and reliability, which makes the concurrent program design one of the most difficult problems of programming~\cite{mckenney2017parallel}.

Traditionally, studies related to concurrent programming focus on more fundamental theoretical questions of designing race-free and lock-free parallel algorithms, asynchronous data structures and synchronisation primitives of a programming language~\cite{ben2006principles}.
Unfortunately, when it comes to real-world concurrent programs, the algorithmic level of abstraction is not enough for guaranteeing their correctness.
The reasons of this fact lie in the code optimisations that both compiler and hardware perform in order to increase performance of the system~\cite{adve1996shared}.

As an example, consider a shared memory concurrent system that executes two parallel processes as it is described in Figure~\ref{simple_wmm_x86} (such little examples that illustrate specific behaviour of a concurrent execution environment are called \textit{litmus tests}).
The process~\texttt{P0} writes the value~\texttt{1} to the shared variable~\texttt{x} and reads a value of the shared variable~\texttt{y}, and the process~\texttt{P1} writes the~\texttt{y} and reads the~\texttt{x}.
Considering all interleavings of the processes, one could expect the final state to be one of the following:
\begin{itemize}[noitemsep,topsep=0pt]
\item \texttt{(0:EAX=0,\,1:EBX=1)},
\item \texttt{(0:EAX=1,\,1:EBX=0)},
\item \texttt{(0:EAX=1,\,1:EBX=1)}.
\end{itemize}
%TODO!!!!!!!!!!!!!!!!!!! cite "mcKenney - Memory Barriers a Hardware View for Software Hackers.pdf" -- a table for multiple architectures

%As the program specifies the write to~\texttt{x} to precede the read of~\texttt{y} in process~\texttt{P0}, one could expect the final state `\texttt{(0:EAX=0~/\textbackslash~1:EAX=0)}' to be unreachable.
However, on the x86 architecture the state `\texttt{(0:EAX=0,\,1:EAX=0)}' is also reachable as the processors may cache the write to shared memory into their local write buffers, so that the write does not immediately become visible by processes running on other cores.
This behaviour is known as \textit{store buffering~(SB)}.

\begin{figure}[t]
\centering
\ttfamily
\begin{tabular}{ |l|l| }
\hline
\multicolumn{2}{|l|}{ \{ x=0; y=0; \}} \tabularnewline \hline
P0 & P1 \\ \hline
MOV [x],1 & MOV [y],1 \\
MOV EAX,[y] & MOV EBX,[x] \\
\hline
%\multicolumn{2}{|l|}{locations [x;y;]} \tabularnewline
\multicolumn{2}{|l|}{exists (0:EAX=0,~1:EBX=0)} \tabularnewline
\hline
\multicolumn{2}{|l|}{x86: allow} \tabularnewline
\hline
\end{tabular}
\caption{Store buffering: A litmus test illustrating the write-read reordering allowed by the x86-TSO memory model}
\label{simple_wmm_x86}
\end{figure}

%In the example, the write `\texttt{MOV~[x],1}' performed by process \texttt{P0} stores value~\texttt{1} to the shared variable~\texttt{[x]} into the write buffer of process~\texttt{P0}.
%Meanwhile, the write cache of the process \texttt{P1} may not have an updated version of the variable~\texttt{[x]}, neither the main memory, so that the read `\texttt{MOV~EBX,[x]}' performed in the process~\texttt{P1} may read the initial value~\texttt{0} even if this variable has been already updated in the other process.
%The automation of the memory model-aware program analysis constitutes the main research interest of the current thesis.

The formal way to define the semantics of memory operations and synchronisation primitives of a parallel execution environment is to define its \textit{memory model}.
%The behaviour of a concurrent execution environment is described by a formal memory model
%The study of concurrency of a specific platform concerns on defining
%and are widely used to illustrate its behaviour.
%Properties of an execution environment can be described by a formal \textit{memory model}
%The issues like one discussed above have led to the need for formalisation of semantics of memory operations within different concurrent architectures defined by \textit{memory models}.
There are two main types of formal memory models.
Models of the first type characterise the behaviour of the system (its \textit{operational semantics}) in terms of the abstract machine executing the code, as it was done for the SB example above.
Models of the second type define the \textit{axiomatic semantics} of the system by specifying the set of assertions over states of the program.
%There are two ways to define a memory model: by specifying its \textit{operational semantics} (in terms of an abstract machine executing the code, as it was done for the SB example above) or by defining its \textit{axiomatic semantics} (in terms of assertions over states of the program formed by sequences of operations).
Although the former type of memory models may be easier to describe and interpret, existing numerous formal verification tools and methods address the research towards axiomatic memory models.

The first memory model for a concurrent system was formulated by Leslie Lamport back in 1979~\cite{lamport1979make}.
This memory model, called the \textit{sequential consistency (SC)}, allows only those executions that produce the same result as if the operations had been executed in an interleaved fashion in a single process%
%
\footnote{In order to prescind from the implementation details while discussing the memory models theory, we avoid the use of the software-specific term \textit{thread} and the hardware-specific term \textit{processor}. Instead, we adhere the terminology of the theory of concurrency employed by Ben-Ari~\cite{ben2006principles} by naming a concurrent piece of code the \textit{process}.}. %
%
This means that the order of operations executed by a process is strictly defined by the program (the code) it executes.
The SC model does require the write to a shared variable performed in one process to become visible by all other processes \textit{instantly} as each process writes directly to the shared memory, without local buffering.
Another important requirement of the SC memory model is that it forbids reordering of memory operations within a single process (the order is strictly defined by the program).
Originally, the operational semantics was defined for the SC model, however there exist axiomatic specifications for it~\cite{mansky2015axiomatic}.

The SC model is considered to be a \textit{strong memory model} in the sense that it provides firm guarantees regarding the ordering and effect of memory operations.
Different relaxations of the SC model lead to \textit{weak memory models~(WMMs)} that specify how processes interact through the shared memory, when a write becomes visible to processes running on other cores, and what value a read operation can get.
Thus, WMMs serve as set of guarantees made by designers of execution environment (hardware, programming language, compiler, database, operation system, etc.) to programmers on which behaviours of their concurrent code they can rely on.


\section{Related work}
\label{ch:intro:related}

Research on weak memory models firstly aims to \textit{formalise} a formal approach of understanding programs with respect to weak memory models, which is \textit{systematic}, \textit{sound} and \textit{complete}.
One of the most well-known frameworks for weak memory model-aware analysis was formalised by J. Alglave in~2010~\cite{alglave2010shared}.
It is the event-based non-deterministic model without global time (see Section~\ref{ch:wmm:event} for details).

In addition to developing the theoretical basis, researchers work on extracting the memory models for hardware architectures from existing implementations or from the specifications, which are written in natural language and thus suffer from ambiguities and incompleteness.
Over the last decade, memory models have been defined for most mainstream multiprocessor architectures, such as x86-TSO and Sparc-TSO (for \textit{Total Store Order}) model for x86 and Sparc architectures~\cite{owens2009better}, much more relaxed memory model for Power and ARM architectures~\cite{alglave2009semantics,sarkar2011understanding, alglave2014herding}, etc. %, or Alpha architecture~\cite{alphaWMM}~, and others.
There are projects for validating hardware architectures wrt. a memory model as well, e.g.~\cite{lustig2014pipecheck,lustig2016coatcheck}.

%TODO: litmus test synthesis 
%1) http://research.nvidia.com/sites/default/files/pubs/2017-04_Automated-Synthesis-of//ASPLOS_2017_Memory_Model_Verification.pdf

Most modern high-level programming languages rely on relaxed memory model as well.
Thus, the memory model for Java is based on the \textit{happens-before} principle~\cite{lamport1978time} and was introduced in J2SE 5.0 in 2004~\cite{manson2005java}.
The transformations valid for the Java memory model were discussed in~\cite{sevcik2009program}. 
A weak memory model for C and C++ is based on the relation \textit{strongly happens-before} and was introduced in the C++17 standard~\cite{batty2011mathematizing}.
Before that, the C++11 standard~\cite{iso2012iec} proposed the set of hardware-independent synchronisation fences and atomic operations.
Weak memory models are being formalised for even more abstract software environments, the notable project in this area is the project on formalising the Linux kernel memory model, which is being actively developing these days~\cite{alglave2018frightening, kernel1}.

Furthermore, there exists a wide range of tools that perform memory model-aware analysis.
\begin{itemize}[noitemsep,topsep=0pt]

\item A state-of-the-art tool is \tool{diy} (\textit{do it yourself}), developed by researchers from INRIA institute, France and University of Cambridge, UK.
The \tool{diy}%
%
\footnote{The \tool{diy} project web site: \url{http://diy.inria.fr/}} %
%
is a software suite for designing and testing weak memory models.
It is firstly released back in 2010, and since that time it remained to be the only tool for testing weak memory models.
The \tool{diy} consists of several modules:
the litmus tests generators \tool{diy}, \tool{diycross} and \tool{diyone},
the litmus test concrete executor \tool{litmus} that runs tests on a physical machine and collects its behaviours,
and the weak memory model simulator \tool{herd} that implements reachability analysis for exploring states reachable under the specified~WMM.

\item There exist tools that perform the weak memory model-aware program verification and model checking.
The notable examples are
%the tool \tool{mmchecker} for model-checking the C\# code~\cite{huynh2006memory}
the stateless model checkers \tool{RCMC}~\cite{kokologiannakis2017effective}, \tool{CHESS}~\cite{musuvathi2008fair} and \tool{Nidhugg}~\cite{abdulla2017stateless},
the tool \tool{Trencher} for checking programs against the TSO memory model~\cite{bouajjani2013checking}, 
the tool \porthos{} for portability analysis~\cite{Porthos17a},

\item Some tools tackle the problem of automated synthesis of the synchronisation primitives, such as
the automatic fence insertion tool \tool{musketeer}~\cite{alglave2014don}, and
the automatic verification and fence inference tool \tool{blender}~\cite{kuperstein2011partial}.

\item Some other tools perform static instrumentation of concurrent C programs and encode the WMM into the program representation so that it can be model-checked by standard tools.
The examples are
%TODO: cite the 'wmm-general/Thread-Modular Static Analysis for Relaxed Memory Models.pdf'
the instrumenting compiler \tool{goto-cc} which is a part of \tool{CBMC} model checker~\cite{kroening2014cbmc},
the tool that performs the sequentialisation of concurrent programs~\cite{alglave2013software},
the tool \tool{Weak2SC} for producing program descriptions which can be fed into standard model checking tools (such as \tool{SPIN}~\cite{holzmann1997model} or \tool{NuSMV}~\cite{cimatti2000nusmv}) for performing memory model-aware analysis~\cite{travkin2016verification}.
\end{itemize}

%Although weak memory studies is rather young research area, there exist frameworks and tools for exploring WMMs and examining simple programs with respect to them.

All the tools listed above consider only a single memory model, however, in real life we face serious engineering problems involving necessity to model more than one execution environment.
One of these problems is the \textit{portability} of the program from one hardware architecture to another.
A program written in a high-level language is then compiled for different hardware.
Even if all the compiler optimisations were disabled (which is rare case nowadays), the behaviour of two compiled versions of the same program may differ due to differences between hardware memory models.
As the result, a program compiled under the platform $\mathcal{T}$ can reach states that are unreachable on the platform $\mathcal{S}$, which is a \textit{portability bug} from the source platform $\mathcal{S}$ to the target platform $\mathcal{T}$~\cite{Porthos17a}.

The very first tool that performs the WMM-aware portability analysis is \porthos%
%
\footnote{The \porthos{} project web site: \url{http://github.com/hernanponcedeleon/Dat3M}} %
%
introduced in April~2017~\cite{Porthos17b}.
This tool reduces described problem to a bounded reachability problem, which can be solved via an SMT solver.
This approach allows to capture symbolically the semantics of analysing program and both weak memory models into a single SMT-formula, augmented by the reachability assertion.
As most modern SMT solvers are efficient enough to be able to operate the state space of size millions of variables bounded by millions of constraints~(\cite{malik2009boolean}), the used method can be applicable in solving the real-world problems.
% TODO: search space size: https://courses.cs.washington.edu/courses/csep573/11wi/lectures/ashish-satsolvers.pdf slide 

%In the work~\cite{Porthos17b}, the SMT-based approach was defined for analysing the portability of a program from one hardware architechture to another, which is defined as ``an execution that is consistent with the target but inconsistent with the source memory model". Although encoding the control-flow and the data-flow of a program into an SMT-formula seems to be a tril problem of symbolic execution, encoding of the weak memory model is more tedious. The reason is that some relations of WMMs are defined as mutually-recursive and need to be linearised in order to be encoded into an equivalent logical formula.

%"A portability bug is an execution that is consistent with the target but inconsistent with the source memory model. We capture this alternation with a single existential query. Consistency is specified in terms of acyclicity (and irreflexivity) of relations. Hence, an execution is inconsistent if a derived relation of the (source) memory model contains a cycle (or is not irreflexive)."

% PROBLEM STATEMENT: MORE CLEAR: 
% 1) Definition of research scope and goals 
% 2) "- References throw light on the topic from a variety of perspectives"
%    "-The thesis combines the cited  works and empirical data consistently and clearly"
%    "- Thesis demonstrates critIcal evaluation of existing knowledge"
% 3) Research Methods
%    "- Appropriate methods have been used in a well-founded manner. Reliability and validity or  trustworthiness of the study/results are evaluated."
% 4) Results and contribution 
%    results + justification
%    Goals and results must match
%    the results must be based on independent research , not on the references.
%    "significance for the field of research or industry in question, and contribution to knowledge"
%    "Cited works must be evaluated critically"
%    "Conclusions SHOULD NOT BE drawn JUST appropriately from the material"

\section{Task specification}
\label{ch:intro:task}

The current work aims to rework the proof-of-concept tool \textit{\porthos{}} by extending the input language, which currently represents the minimum subset of C, and revising the general architecture of the tool in order to enhance performance, extensibility, reliability and maintainability.
One of the directions of development was the ability to process the \textit{kernel litmus tests} written in C~\cite{mckenney2017wg21}.
For that, in addition to supporting the C syntax (augmented by litmus-style definitions such as initialisation or assertion statements), the tool must be able to recognise kernel-specific functions and macros (such as the macro \texttt{READ\_ONCE} that guarantees memory read) and be easily extensible for defining new functions (have the separate module for the purpose of a knowledge base).
%interprocedural analysis

As the general architecture and almost all components of \porthos{} were to be redesigned, the tool received a new name \textit{\porthos[2]}\,%
%
\footnote{Hereinafter with the names \textit{\porthos{}} and \textit{\porthos[1]} we refer to the tool \porthos{} of version 1, whereas the new implementation of \porthos{} is called \textit{\porthos[2]}.}. %
%
Considering the enhancements of the architectural design, \porthos[2] represents a generalised framework for SMT-based memory model-aware analysis, which can not only perform the reachability and portability analysis, but serve as a basis for other kinds of static analysis of concurrent programs.


\section{Thesis structure}
\label{ch:intro:structure}

The thesis is organised as following.
Chapter~\ref{ch:wmm} gives a general view on the weak memory model-aware analysis.
Chapter~\ref{ch:enc} examines the portability analysis as an bounded reachability problem that can be encoded into an SMT-formula in order to be solved automatically by an SMT~solver.
Chapter~\ref{ch:impl} delves into the description of architectural solutions and implementation details of the \porthos[2] framework.
Chapter~\ref{ch:eval} provides an example of the portability analysis via \porthos[2] and presents some evaluation results comparing to other tools. %TODO: check if it's true
Chapter~\ref{ch:summary} summarises results of the work and proposes future work directions. %TODO: check if it's true
