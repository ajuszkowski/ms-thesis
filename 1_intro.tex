\chapter{Introduction}
\label{ch:intro}

%TODO: rename mousquetaires: http://www.cprover.org/wmm/musketeer/ !!! (+reference this project: "Don’t sit on the fence".pdf in alglave/ )
% TODO: see how musketeer deals with recursive calls: http://www.cprover.org/wmm/musketeer/rec.pdf

Most modern computer systems contain large parts that operate concurrently. Though parallelisation of the system can improve its performance drastically, it opens numerous of problems connected to correctness, robustness and reliability, which makes the concurrent program design one of the most difficult problems of programming~\cite{mckenney2017parallel}.
% Most articles, presentations and books on concurrent programming start with words how hard it is.

Traditionally, studies related to concurrent programming concern on more fundamental theoretical questions of designing race-free and lock-free parallel algorithms, asynchronous data structures and synchronisation primitives of a programming language.
Unfortunately, when it comes to the real-world concurrent programs, the algorithmic level of abstraction is not enough for guaranteeing their properties of correctness and reliability.
The reasons of this fact lie in the code optimisations that both compiler and hardware perform in order to increase performance as much as possible~\cite{adve1996shared}.
For instance, Figure~\ref{simple_wmm_x86} provides simple example of reachability of the state `\texttt{(0:EAX=0~/\textbackslash~1:EAX=0)}' on the x86 architecture (such little examples that illustrate specific behaviour of a WMM are called \textit{litmus tests}).
%reordering of memory access instructions within single process allowed by the x86-TSO weak memory model, which potentially breaks the program logic. 
This state is allowed because in the x86 architecture each processor may cache the write to shared memory variable into its local write buffer, so that they do not immediately become visible by processes running on other cores.
In the example, the write `\texttt{MOV~[x],1}' performed by process \texttt{P0} stores value~\texttt{1} to the shared variable~\texttt{[x]} into the write buffer of process~\texttt{P0}.
Meanwhile, the write cache of the process \texttt{P1} may not have an updated version of the variable~\texttt{[x]}, neither the main memory, so that the read `\texttt{MOV~EBX,[x]}' performed in the process~\texttt{P1} may read the initial value~\texttt{0} even if this variable has been already updated in the other thread.
These problems have lead to the need for formalisation of semantics of memory operations within different concurrent architectures defined by \textit{weak memory models (WMMs)}.

\begin{figure}
\centering
\ttfamily
\begin{tabular}{ |l|l| }
\hline
\multicolumn{2}{|l|}{ \{ x=0; y=0; \}} \tabularnewline \hline
P0 & P1 \\ \hline
MOV [x],1 & MOV [y],1 \\
MOV EAX,[y] & MOV EAX,[x] \\
\hline
%\multicolumn{2}{|l|}{locations [x;y;]} \tabularnewline
\multicolumn{2}{|l|}{exists (0:EAX=0~/\textbackslash~1:EAX=0)} \tabularnewline
\hline
\multicolumn{2}{|l|}{x86-TSO: allow} \tabularnewline
\hline

\end{tabular}
\caption{Store buffering (SB): a litmus test on write-read reordering allowed under the x86-TSO and forbidden under the SC memory model}
\label{simple_wmm_x86}
\end{figure}

Research of weak memory models firstly aims to \textit{formalise} develop the formal approach of understanding programs with respect to weak memory models which is systematic, sound and complete. The first (and so far the only) such a framework was formalised by J. Alglave in 2010~\cite{alglave2010shared}.
In addition to developing rather theoretical basis, researchers work on extracting the WMMs for hardware architectures from existing implementations of from their specifications, which are written in natural language and thus suffer from ambiguities and incompleteness.
Over last decade the memory models have been defined for most mainstream multiprocessor architectures, such as x86-TSO and Sparc-TSO (for \textit{Total Store Order}) model for x86 and Sparc architecture formalised in 2009~\cite{owens2009better}, much more relaxed memory model for Power and ARM architectures~\cite{alglave2009semantics,sarkar2011understanding, alglave2014herding}, and others.
There are projects for validating hardware architectures wrt. a memory model, e.g.~\cite{lustig2014pipecheck,lustig2016coatcheck}.

Most modern high-level programming languages rely on relaxed memory model as well.
Thus, the memory model for Java is based on the \textit{happens-before} principle~\cite{lamport1978time}, it was introduced in J2SE 5.0 in 2004~\cite{manson2005java}; the C++11 standard~\cite{iso2012iec} has introduced the set of hardware-independent synchronisation fences and atomic operations, whenever the C++17 memory model~\cite{batty2011mathematizing} is based on the relation \textit{strongly happens-before}. Weak memory are being formalised for even more abstract software environments, the notable project in this area is the project on formalising the Linux kernel memory model, which is being actively developing these days~\cite{kernel1}.

Furthermore, there exists a wide range of tools that perform program verification wrt. memory models (e.g., \cite{alglave2013software}, the tool \tool{Trencher} for checking programs against the TSO memory model~\cite{bouajjani2013checking}, the tool \porthos{} for portability analysis~\cite{Porthos17SAS})
, weak memory aware model checking (e.g., the model-checker \tool{blender}~\cite{kuperstein2011partial}, the tool \tool{mmchecker} for model-checking the C\# code~\cite{huynh2006memory})
, synchronisation synthesis (e.g., the automatic fence insertion tool \tool{musketeer}~\cite{alglave2014don})
, static instrumentation of concurrent C programs for further analysis wrt. weak memory model (e.g., the instrumenting compiler \tool{goto-cc} which is a part of \tool{CBMC} model checker~\cite{kroening2014cbmc}, the tool \tool{Weak2SC} for producing program descriptions which can be fed into standard model checking tools for performing memory model-aware analysis~\cite{travkin2016verification})
, and other.

%experiments with kernel?

The first memory model for concurrent systems was formulated by Leslie Lamport back in 1979~\cite{lamport1979make}. This memory model, called the \textit{sequential consistency (SC)}, allows only those executions (interleavings) that produce the same result as if the operations had been executed by single process. This means that the order of operations executed by a process is strictly defined by the program it executes. The SC model does requires the write to a shared variable performed in one process to become visible by all other processes not instantly, but simultaneously. This means each process communicates to the shared memory directly, without local buffering. Another important requirement of SC memory model is that it forbids memory operations reordering within single process (the order is strictly defined by the program).

The SC model is considered to be the strong memory model in the sense that it provides strong guarantees regarding the ordering and caused effect of memory operations. Different relaxations of this model lead to the class of \textit{weak memory models~(WMM)}.
%while preserving consistency
They specify how threads interact through shared memory, when a write becomes visible to other threads and what value a read can return. 
Therefore, WMMs serve as set of guarantees made by designers of execution environment (hardware, programming language, compiler, database, operation system, etc.) to programmers on which behaviours of their concurrent code they may expect. 

Although weak memory studies is rather young research area, there exist frameworks and tools for exploring WMMs and examining simple programs with respect to the them. The state-of-the-art tool is \tool{diy} (for \textit{do it yourself}), developed by the researchers from INRIA institute, France and University of Cambridge, UK.
The \tool{diy}\footnote{Project web site: \url{http://diy.inria.fr/}} is a software suite for designing and testing weak memory models. It is firstly released back in 2010, and since that time it remained to be the only tool for testing weak memory models. The \tool{diy} consists of several modules: the litmus tests generators \tool{diy}, \tool{diycross} and \tool{diyone}, the litmus tests concrete executor \tool{litmus} that runs tests on a physical machine while collecting its behaviours, and the weak memory models simulator \tool{herd} that implements reachability analysis for exploring states reachable under specified~WMM.

All the \tool{diy} tools work only with single memory model, however, in real life we face serious engineering problems involving necessity to model more than one execution environment. One of these problems is the \textit{portability} of the program from one hardware architecture to another. A program written in a high-level language is then compiled for different hardware. Even if all the compiler optimisations were disabled (which is rare case nowadays), the behaviour of two compiled versions of the same program may differ due to differences between hardware memory models.
As the result, a program compiled under the platforms $T$ can reach states that are unreachable on the platform $S$, which is a \textit{portability bug} from the source platform $S$ to the target platform $T$~\cite{Porthos17SAS}.

The first tool that performs the WMM-aware portability analysis is \porthos%
\footnote{Project web site: \url{http://github.com/hernanponcedeleon/PORTHOS}}%
introduced in April~2017~\cite{Porthos17CoRR}. This tool reduces described problem to a bounded reachability problem, which can be solved with help of an SMT-solver. This approach allows to capture symbolically the semantics of analysing program and both weak memory models into single SMT-formula, augmented by the reachability assertion. As most modern SMT-solvers are efficient enough to be able to operate the state space of size millions of variables bounded by millions of constraints~(\cite{malik2009boolean}), the used method can be applicable in solving the real-world problems.
% TODO: search space size: https://courses.cs.washington.edu/courses/csep573/11wi/lectures/ashish-satsolvers.pdf slide 

%In the work~\cite{Porthos17CoRR}, the SMT-based approach was defined for analysing the portability of a program from one hardware architechture to another, which is defined as ``an execution that is consistent with the target but inconsistent with the source memory model". Although encoding the control-flow and the data-flow of a program into an SMT-formula seems to be a trivial problem of symbolic execution, encoding of the weak memory model is more tedious. The reason is that some relations of WMMs are defined as mutually-recursive and need to be linearised in order to be encoded into an equivalent logical formula.

%"A portability bug is an execution that is consistent with the target but inconsistent with the source memory model. We capture this alternation with a single existential query. Consistency is specified in terms of acyclicity (and irreflexivity) of relations. Hence, an execution is inconsistent if a derived relation of the (source) memory model contains a cycle (or is not irreflexive)."

% PROBLEM STATEMENT: MORE CLEAR: 
% 1) Definition of research scope and goals 
% 2) "- References throw light on the topic from a variety of perspectives"
%    "-The thesis combines the cited  works and empirical data consistently and clearly"
%    "- Thesis demonstrates critIcal evaluation of existing knowledge"
% 3) Research Methods
%    "- Appropriate methods have been used in a well-founded manner. Reliability and validity or  trustworthiness of the study/results are evaluated."
% 4) Results and contribution 
%    results + justification
%    Goals and results must match
%    the results must be based on independent research , not on the references.
%    "significance for the field of research or industry in question, and contribution to knowledge"
%    "Cited works must be evaluated critically"
%    "Conclusions SHOULD NOT BE drawn JUST appropriately from the material"

Current work aims to rework the proof-of-concept tool \porthos{} by extending the input language, which currently represents the minimum subset of C, and revising the general architecture of the tool in order to enhance performance, extensibility, reliability and maintainability.
%interprocedural analysis
As the general architecture and almost all components of \porthos{} have been redesigned, the tool received a new name -- \porthos[2]%
\footnote{Hereinafter with the name `\porthos' we refer to the tool \porthos{} version 1 (also addressed as \porthos[1]), whereas the new version of \porthos{} is called \porthos[2].} %
. Considering the enhancements of the architecture, \porthos[2] represents a generalised framework for SMT-based memory model-aware analysis, which can not only perform the portability analysis, but can serve as a basis for other kinds of static code analysis.


\section{Thesis structure}
\label{ch:intro:structure}

The thesis is organised as following. Chapter~\ref{ch:wmm} gives a general view on the weak memory model-aware analysis. Chapter~...

% TODO: Add intro to the bounded reachability analysis using SAT -- ? 

% TODO: other tools for bounded model-checking: 
%Clarke's goto-cc : \cite{clarke2004tool} : http://www.cprover.org/goto-cc/
% (perhaps, goto-cc is a part of cprover) : cprover https://www.cprover.org/cbmc/doc/manual.pdf
%cbmc http://www.cprover.org/cbmc/